\documentclass[12pt,a4paper]{article}
\usepackage{graphicx}
\usepackage{wrapfig}
\usepackage{textcomp}

\title{Praktikum Physik - Leitercharakteristiken}
\author{Simon Marti, Patricia Schwab, Mirco Kocher}
\date{20.04.2012}

\parindent=0pt 
\begin{document}
\maketitle

\section*{Ziel}


\section*{Motivation}


\section*{Theorie}


\section*{Experiment}

\subsection*{Aufbau und Ablauf}

\subsection*{Rohdaten}

\subsubsection*{Drahtwiederstand}

\subsubsection*{Metallfadenlampe}

\subsubsection*{Kohlenfadenlampe}

\subsubsection*{Silizium-Diode}

\subsubsection*{Zenerdiode}

\subsubsection*{Glimmstabilisator}


\subsection*{Auswertung}

\subsubsection*{Drahtwiederstand}

\subsubsection*{Metallfadenlampe}

\subsubsection*{Kohlenfadenlampe}

\subsubsection*{Silizium-Diode}

\subsubsection*{Zenerdiode}

\subsubsection*{Glimmstabilisator}


\subsection*{Diskussion}

\newpage
\section*{Theorieaufgaben}

\subsection*{Aufgabe 2}
\subsubsection*{Knotenregel}
Die Summe der zu- und wegfliessenden Str\"omen in einem Knoten ist Null, d.h.
\[ \sum_kl_k = 0 \]
Die Knotenregel folgt aus der Ladungserhaltung.

\subsubsection*{Maschenregel}
Die Summe aller Spannungsabf\"alle an den einzelnen Elementen, aus denen eine Masche besteht, ist Null, d.h.
\[ \sum_kU_k = 0\]
Die Maschenregel folgt aus der Energieerhaltung.

\subsection*{Aufgabe 3}
\begin{eqnarray*}
R & = & \frac{38}{g} k\Omega \\
I & = & \frac{U}{R} = \frac{9}{38000} \cdot U_0
\end{eqnarray*}

Aus der Knotenregel
\begin{eqnarray*}
I_5 & = & I_{23} + I_4 \hspace{20pt} ( I_{23} = I_2 = I_3 ) \\
I_1 & = & I_3 + I_4 \hspace{20pt} ( I = I_1 = I_5 )
\end{eqnarray*}
folgt:
\begin{eqnarray*}
U_5 & = & I \cdot R_5 = \frac{9}{38000} \cdot U_0 \cdot 10^3 = \frac{9}{38} \cdot U_0 = I \cdot R_1 = U_1 \\
I_{23} & = & \frac{U_5}{R_{23}} = \frac{9}{38} \cdot U_0 \cdot \frac{1}{5000} = I_2 = I_3 \\
\Rightarrow & U_2 & = R_2 \cdot I_2 = \frac{9}{38} \cdot \frac{2000}{5000} \cdot U_0 = \frac{9}{95} \cdot U_0 \\
& U_3 & = R_3 \cdot I_3 = \frac{9}{38} \cdot \frac{3000}{5000} \cdot U_0 = \frac{27}{190} \cdot U_0 \\
I_4 & = & \frac{U_5}{R_4} \\
U_4 & = & I_4 \cdot R_4 = \frac{U_5}{R_4} = U_5 = \frac{9}{38} \cdot U_0
\end{eqnarray*}
\"Uberpr\"ufung:
\[ \sum_{i=1}^5 U_i = 3 \cdot \frac{9}{38} \cdot U_0 + \frac{9}{95} \cdot U_0 = U_0 \]

\subsection*{Aufgabe 5}
$R_i = 1\Omega$, Messbereich 0-1A, gew\"unschter Bereich 0-10A
\begin{eqnarray*}
I_n & = & 10 - 1 = 9\mbox{A (Strom durch den Nebenwiederstand)} \\
R_n & = & \frac{U_max}{I_n} = \frac{1}{9}  \Omega
\end{eqnarray*}
$R_n$ muss parallel geschaltet werden.
\begin{eqnarray*}
R_{Ges} & = & \left( \frac{1}{R_i} + \frac{1}{R_n} \right) ^{-1} = \frac{1}{10}\Omega \\
R_{Ges} \cdot I & = & U_{max} \\
\frac{1}{10} \cdot 10 & = & 1 \mbox{V}
\end{eqnarray*}

\end{document}