\documentclass[12pt,a4paper]{article}
\usepackage{graphicx}
\usepackage{wrapfig}

\title{Praktikum Physik - Erzwungene Schwingungen}
\author{Simon Marti, Patricia Schwab, Mirco Kocher}
\date{09.03.2012}

\parindent=0pt 
\begin{document}
\maketitle

%%
% Ziel
%%
\section*{Ziel}
Messung der Eigenfrequenz der freien Schwingung bei unterschiedlichen D\"ampfungen und Bestimmung der D\"ampfungskonstante.

%%
% Motivation
%%
\section*{Motivation}
Der Versuch der erwungenen Schwingungen verschafft einen guten \"Uberblick \"uber das Gebiet der Differentalgleichungen. Ausserdem k\"onnen die gemessenen Daten graphisch auf logarithmischem Papier dargestellt werden. 

%%
% Theorie
%%
\section*{Theorie}
Periodendauer
\begin{equation}
T = \frac{Perioden}{Sekunden} \mbox [{s}^{-1}]
\end{equation}
Alpha
\begin{equation}
\alpha = \frac{-T\cdot\log\left( \frac{erster Wert}{letzter Wert}\right)}{Anzahl Werte} \mbox [{whatever}]
\end{equation}
Winkelgeschwindigkeit $\omega_0$
\begin{equation}
\omega_0 = 2\pi \cdot T \mbox[{rad \cdot s}^{-1}]
\end{equation}
Resonanzfrequenz
\begin{equation}
\Omega_R = \sqrt{\omega_0^2 - 2\alpha^2} \mbox [{s}^{-1}]
\end{equation}

\newpage

%%
% Experiment 1
%%
\section*{Experiment I}

% Aufbau und Ablauf
\subsection*{Aufbau und Ablauf}
Die Apparatur besteht aus einer vertikal stehenden Drehscheibe mit r\"ucktreibender Spiralfeder dessen Auslenkung an einer Skala abgelesen werden kann. Zwei Drahtspulen befinden sich beidseitig der Scheibe und wirken als Wirbelstrombremse deren St\"arke durch die Stromzufuhr geregelt werden kann. Ein Motor mit kontinuierlich verstellbarer Drehfrequenz liefert ein periodisches, externes Moment indem das feste Ende der Spiralfeder bewegt wird. Die Amplitude des Pendels wird jeweils von Patricia an der eingebauten Skala abgelesen, die Frequenz des Pendels und des Motors wird bestimmt indem Simon mir einer Stoppuhr zehn Perioden misst.

Beim ersten Experiment wird die Amplitude des Pendels in mehreren aufeinanderfolgenden Schwingungszyklen für verschiedene Dämpfungen gemessen. Das Pendel wurde jeweils bis zum Wert 9 auf der Skala ausgelenkt.

% Rohdaten
\subsection*{Rohdaten}
\subsubsection*{Ohne D\"ampfung}
Erste Messung

\vspace{3pt}
\begin{tabular}{|l|l|l|l|l|l|l|l|l|l|}
\hline
$A_{0}$&$A_{1}$&$A_{2}$&$A_{3}$&$A_{4}$&$A_{5}$&$A_{6}$&$A_{7}$&$A_{8}$&$A_{9}$\\
\hline
9.0&8.8&8.7&8.5&8.4&8.2&8.1&8.0&7.8&7.7\\
\hline
\hline
$A_{10}$&$A_{11}$&$A_{12}$&$A_{13}$&$A_{14}$&$A_{15}$&$A_{16}$&$A_{17}$&$A_{18}$&$A_{19}$\\
\hline
7.6&7.4&7.3&7.2&7.0&6.9&6.7&6.6&6.5&6.3\\
\hline
\hline
$A_{20}$&$A_{21}$&$A_{22}$&$A_{23}$&$A_{24}$&$A_{25}$&$A_{26}$&$A_{27}$&$A_{28}$&$A_{29}$\\
\hline
6.2&6.1&5.9&5.8&5.7&5.5&5.4&5.3&5.1&5.0\\
\hline
\end{tabular}

\vspace{10pt}
Zweite Messung

\vspace{3pt}
\begin{tabular}{|l|l|l|l|l|l|l|l|l|l|}
\hline
$A_{0}$&$A_{1}$&$A_{2}$&$A_{3}$&$A_{4}$&$A_{5}$&$A_{6}$&$A_{7}$&$A_{8}$&$A_{9}$\\
\hline
9.0&8.9&8.7&8.5&8.4&8.3&8.1&7.9&7.8&7.6\\
\hline
\hline
$A_{10}$&$A_{11}$&$A_{12}$&$A_{13}$&$A_{14}$&$A_{15}$&$A_{16}$&$A_{17}$&$A_{18}$&$A_{19}$\\
\hline
7.5&7.3&7.2&7.0&6.9&6.7&6.6&6.5&6.4&6.2\\
\hline
\hline
$A_{20}$&$A_{21}$&$A_{22}$&$A_{23}$&$A_{24}$&$A_{25}$&$A_{26}$&$A_{27}$&$A_{28}$&$A_{29}$\\
\hline
6.1&5.9&5.8&5.7&5.5&5.4&5.3&5.2&5.0&4.8\\
\hline
\end{tabular}

\subsubsection*{Mit D\"ampfung 0.45A}
Erste Messung

\vspace{3pt}
\begin{tabular}{|l|l|l|l|l|l|l|l|l|l|}
\hline
$A_{0}$&$A_{1}$&$A_{2}$&$A_{3}$&$A_{4}$&$A_{5}$&$A_{6}$&$A_{7}$&$A_{8}$&$A_{9}$\\
\hline
9.0&7.1&5.7&4.5&3.5&2.7&2.2&1.7&1.4&1.0\\
\hline
\end{tabular}
\vspace{10pt}

Zweite Messung

\vspace{3pt}
\begin{tabular}{|l|l|l|l|l|l|l|l|l|l|}
\hline
$A_{0}$&$A_{1}$&$A_{2}$&$A_{3}$&$A_{4}$&$A_{5}$&$A_{6}$&$A_{7}$&$A_{8}$&$A_{9}$\\
\hline
9.0&7.2&5.8&4.5&3.5&2.7&2.1&1.6&1.2&0.9\\
\hline
\end{tabular}

\subsubsection*{Mit D\"ampfung 0.6A}
Erste Messung

\begin{tabular}{|l|l|l|l|l|l|l|l|l|l|}
\hline
$A_{0}$&$A_{1}$&$A_{2}$&$A_{3}$&$A_{4}$&$A_{5}$&$A_{6}$&$A_{7}$&$A_{8}$&$A_{9}$\\
\hline
9.0&6.0&4.0&2.6&1.7&1.1&0.6&0.4&0.2&0.1\\
\hline
\end{tabular}
\vspace{10pt}

Zweite Messung

\vspace{3pt}
\begin{tabular}{|l|l|l|l|l|l|l|l|l|l|}
\hline
$A_{0}$&$A_{1}$&$A_{2}$&$A_{3}$&$A_{4}$&$A_{5}$&$A_{6}$&$A_{7}$&$A_{8}$&$A_{9}$\\
\hline
9.0&6.0&4.0&2.7&1.8&1.2&0.7&0.4&0.3&0.1\\
\hline
\end{tabular}

% Auswertung
\subsection*{Auswertung}

% Zusammenfassung
\subsection*{Zusammenfassung}


%%
% Diskussion
%%
\section*{Diskussion}


\end{document}