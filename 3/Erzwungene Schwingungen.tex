\documentclass[12pt,a4paper]{article}
\usepackage{graphicx}
\usepackage{wrapfig}

\title{Praktikum Physik - Erzwungene Schwingungen}
\author{Simon Marti, Patricia Schwab, Mirco Kocher}
\date{09.03.2012}

\begin{document}
\maketitle

%%
% Ziel
%%
\section*{Ziel}
Messung der Eigenfrequenz der freien Schwingung bei unterschiedlichen D\"ampfungen und Bestimmung der D\"ampfungskonstante.

%%
% Motivation
%%
\section*{Motivation}
Der Versuch der erwungenen Schwingungen verschafft einen guten \"Uberblick \"uber das Gebiet der Differentalgleichungen. Ausserdem k\"onnen die gemessenen Daten graphisch auf logarithmischem Papier dargestellt werden. 

%%
% Theorie
%%
\section*{Theorie}
Die Periodendauer $T$ ist durch folgende Formel gegeben
\begin{equation}
T = \frac{Sekunden}{Perioden} \mbox [{s}]
\end{equation}
Die D\"ampfungskonstante $\alpha$ l\"asst sich durch folgende Formel absch\"atzen
\begin{equation}
\alpha = \frac{\ln\left( \frac{\varphi(n_{max} \cdot T)}{\varphi(n_{min} \cdot T)}\right)}{-T\cdot (n_{max} - n_{min})} \mbox [{s}^{-1}]
\end{equation}
Die Winkelgeschwindigkeit $\omega_0$ l\"asst sich folgendermassen bestimmen
\begin{equation}
\omega_0 = 2\pi \cdot T \mbox[{rad \cdot s}^{-1}]
\end{equation}
Die Resonanzfrequenz $\omega_0$ ist gegeben durch
\begin{equation}
\Omega_R = \sqrt{\omega_0^2 - 2\alpha^2} \mbox [{s}^{-1}]
\end{equation}

\newpage

%%
% Experiment
%%
\section*{Experiment}

% Aufbau und Ablauf
\subsection*{Aufbau und Ablauf}

% Rohdaten
\subsection*{Rohdaten}

% Fehlerrechnung
\subsection*{Fehlerrechnung}

% Auswertung
\subsection*{Auswertung}

% Zusammenfassung
\subsection*{Zusammenfassung}


%%
% Diskussion
%%
\section*{Diskussion}


\end{document}