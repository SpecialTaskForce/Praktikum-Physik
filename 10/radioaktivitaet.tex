\documentclass[12pt,a4paper]{article}
\usepackage{graphicx}
\usepackage{wrapfig}
\usepackage{textcomp}
\usepackage{multicol}
\usepackage[utf8]{inputenc}

\title{Praktikum Physik - Radioaktivität}
\author{Simon Marti, Patricia Schwab, Mirco Kocher}
\date{18.05.2012}

\begin{document}
\maketitle

\section*{Ziel}
Messung der Abschirmung von radioaktiver Strahlung durch Blei und Aluminium und berechnung der Massenabsorptionskoeffizienten. Zerfallsreihe von Uran $^{236}U$ mit der Nuklidkarte.

\section*{Motivation}
In diesem Experiment wird experimentell nachgewiesen, dass das Absorptionsgesetz seine Berechtigung hat.

\section*{Theorie}
Die Intensit\"at $J$ ist definiert durch Anzahl Zerf\"alle $N$ pro Zeiteinheit $\Delta T$
\begin{equation}
J =\frac{N}{\Delta T}
\end{equation}
Aus dem Absorptionsgesetz mit Absorberdicke $x$ und der Aktivit\"at ohne Abschirmung $J_0$
\begin{equation}
J = J_0 e^{-\mu x}
\end{equation}
folgt f\"ur den Massenabsorptionskoeffizienten $\mu_m$
\begin{equation}
\mu_m = - \frac{ln(\frac{J}{J_0})}{x \varrho}
\end{equation}
Die Poisson-Verteilung mit Mittelwert $\lambda$ ist
\begin{equation}
P(k) = \frac{\lambda^k \cdot e^{-\lambda}}{k!}
\end{equation}

\section*{Rohdaten}
N0 = 4577
t0 = 4:42:29
U0 = 500V


\end{document}