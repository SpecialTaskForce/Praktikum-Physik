\documentclass[12pt,a4paper]{article}
\usepackage{graphicx}
\usepackage{wrapfig}
\usepackage{textcomp}
\usepackage{multicol}
\usepackage[utf8]{inputenc}

\title{Praktikum Physik - Radioaktivität}
\author{Simon Marti, Patricia Schwab, Mirco Kocher}
\date{18.05.2012}

\begin{document}
\maketitle

\section*{Ziel}
Messung der Abschirmung von radioaktiver Strahlung durch Blei und Aluminium und berechnung der Massenabsorptionskoeffizienten. Zerfallsreihe von Uran $^{236}U$ mit der Nuklidkarte.

\section*{Motivation}
In diesem Experiment wird experimentell nachgewiesen, dass das Absorptionsgesetz seine Berechtigung hat.

\section*{Theorie}
\begin{equation}
J = J_0 e^{-\mu x}
\end{equation}

\begin{equation}
J =\frac{N}{\Delta T}
\end{equation}

\begin{equation}
\frac{S_y}{y} = \frac{1}{\sqrt{N}}
\end{equation}

\begin{equation}
t = \frac{N}{J}
\end{equation}

\section*{Rohdaten}
N0 = 4577
t0 = 4:42:29
U0 = 500V


\end{document}