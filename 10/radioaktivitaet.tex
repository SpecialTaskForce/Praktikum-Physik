\documentclass[12pt,a4paper]{article}
\usepackage{graphicx}
\usepackage{wrapfig}
\usepackage{textcomp}
\usepackage{multicol}
\usepackage[utf8]{inputenc}

\title{Praktikum Physik - Radioaktivität}
\author{Simon Marti, Patricia Schwab, Mirco Kocher}
\date{18.05.2012}

\begin{document}
\maketitle

\section*{Ziel}
Messung der Abschirmung von radioaktiver Strahlung durch Blei und Aluminium und Berechnung der Massenabsorptionskoeffizienten. Zerfallsreihe von Uran $^{236}U$ mit der Nuklidkarte bestimmen.

\section*{Motivation}
In diesem Experiment wird experimentell nachgewiesen, dass das Absorptionsgesetz seine Berechtigung hat.

\section*{Theorie}
Die Intensit\"at $J$ ist definiert durch Anzahl Zerf\"alle $N$ pro Zeiteinheit $\Delta T$
\begin{equation}
J =\frac{N}{\Delta T}
\end{equation}
Aus dem Absorptionsgesetz mit Absorberdicke $x$ und der Intensit\"at ohne Abschirmung $J_0$
\begin{equation}
J = J_0 e^{-\mu x}
\end{equation}
folgt f\"ur den Massenabsorptionskoeffizienten $\mu_m$
\begin{equation}
\mu_m = - \frac{ln(\frac{J}{J_0})}{x \varrho}
\end{equation}
Die Poisson-Verteilung mit Mittelwert $\lambda$ ist
\begin{equation}
P(k) = \frac{\lambda^k \cdot e^{-\lambda}}{k!}
\end{equation}


\section*{Aufbau und Ablauf}
Ein radioaktives Präparat wird unter ein Geiger-Müller-Zählrohr gelegt, so dass dazwischen noch dünne Metallplatten zur Abschirmung angebracht werden können. Das Zählrohr ist mit einem elektronischen Zähler verbunden. Für verschiedene Präparate und Abschirmungen wird nun die Impulsrate $A$ gemessen indem Patricia den Zähler eine gewisse Zeit laufen lässt und diese mit einer Stoppuhr misst.


\section*{Aufgabe 1}
\subsection*{Rohdaten}
\[ U_s = 370 \mbox{V} \]

\subsection*{Auswertung}
TODO


\section*{Aufgabe 2}
\subsection*{Rohdaten}
\[ T = 10 \mbox{s} \]
\begin{tabular}{|r|l|r|l|r|l|r|l|r|l|r|l|}
\hline
$t$&$N$&$t$&$N$&$t$&$N$&$t$&$N$&$t$&$N$&$t$&$N$\\
\hline
10&7&180&89&350&178&520&272&690&352&860&455\\
20&13&190&94&360&186&530&277&700&361&870&458\\
30&16&200&99&370&193&540&279&710&365&880&463\\
40&21&210&105&380&195&550&282&720&371&890&466\\
50&25&220&108&390&198&560&287&730&379&900&472\\
60&29&230&115&400&200&570&291&740&383&910&478\\
70&35&240&118&410&208&580&296&750&389&920&479\\
80&42&250&123&420&210&590&304&760&393&930&480\\
90&47&260&130&430&214&600&308&770&397&940&485\\
100&53&270&130&440&220&610&313&780&403&950&492\\
110&56&280&135&450&225&620&315&790&408&960&497\\
120&59&290&143&460&238&630&322&800&414&970&499\\
130&63&300&149&470&245&640&327&810&418&980&503\\
140&69&310&156&480&250&650&332&820&428&990&507\\
150&72&320&162&490&253&660&335&830&433&1000&509\\
160&79&330&170&500&257&670&342&840&439&1010&512\\
170&87&340&172&510&266&680&349&850&449&1020&520\\
\hline
\end{tabular}

\subsection*{Auswertung}
\[ \overline{N} = 5.079 \mbox{T}^{-1} \]
\[ P(N) = \frac{\overline{N}^N}{e^{\overline{N}}N!} \]

\section*{Aufgabe 3}
$^{238}$U $\rightarrow$  $^{234}$Th $\rightarrow$  $^{234}$Pa $\rightarrow$  $^{234}$U $\rightarrow$  $^{230}$Th $\rightarrow$  $^{226}$Ra $\rightarrow$  $^{222}$Rn $\rightarrow$  $^{218}$Po $\rightarrow$  $^{214}$Pb $\rightarrow$  $^{214}$Bi $\rightarrow$  $^{214}$Po $\rightarrow$  $^{210}$Pb $\rightarrow$  $^{210}$Bi $\rightarrow$  $^{210}$Po $\lor$  $^{206}$Ti $\rightarrow$  $^{206}$Pb  \\
Die Farbe gibt die Zerfallsart an. Dabei ist schwarz stabil, gelb ein $\alpha$-Zerfall, lachs ein $\beta^+$-Zerfall und blau ein $\beta^-$-Zerfall. Ausserdem ist die Halbwertszeit und die freigesetzte Energeie bei dem Zerfall aufgef\"uhrt. Die farbigen Fl\"achen sind ein Mass f\"ur die relative H\"aufigkeit der jeweiligen Zerf\"alle.


\section*{Aufgabe 4}
\subsection*{Rohdaten}
\begin{eqnarray*}
N_0 & = & 4577 \\
t_0 & = & 4\mbox{h }42\mbox{min }29\mbox{s} \\
U_0 & = & 500\mbox{V}
\end{eqnarray*}


\section*{Diskussion}

\end{document}