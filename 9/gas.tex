\documentclass[12pt,a4paper]{article}
\usepackage{graphicx}
\usepackage{wrapfig}
\usepackage{textcomp}
\usepackage{multicol}
\usepackage[utf8]{inputenc}

\title{Praktikum Physik - Dichte von Gasen}
\author{Simon Marti, Patricia Schwab, Mirco Kocher}
\date{11.05.2012}


\begin{document}
\maketitle

\section*{Ziel}
TODO Mirco


\section*{Motivation}
TODO Mirco


\section*{Sammlung}
\[ p V = \varrho V R T \]
\[ m_0 = m_{Luft} \frac{P_0}{P_L} \]
\[ m^* g = Gas + kolben - Waagschale - Auftrieb \]

\section*{Theorie}
\[ m^* g = Gas + kolben - Waagschale - Auftrieb \]
Aus
\[ m^*_{Luft} = \rho _{Luft} V_I + m_{Kolben} - m_{Waagschale} - \rho_{Luft} V_A \]
\[ m^*_0 = \rho_{Luft} V_I \frac{P_0}{P_L} + m_{Kolben} - m_{Waagschale} - \rho_{Luft} V_A  \]
folgt
\[ m^*_0 - m^*_{Luft} = \rho_{Luft}V_I \left( \frac{P_0}{P_L}-1\right) \]
\[ \Rightarrow \rho_{Luft} = \frac{m^*_0 - m^*_{Luft}}{V_I \left( \frac{P_0}{P_L}-1\right)} \]
Durch Einsetzen von $\rho_{Luft}$
\[ m_{Kolben} - m_{Waagschale} - \rho_{Luft} V_A = m^*_{Luft} - \rho _{Luft} V_I \]
Nun kann die Dichte bestimmt werden
\[ \rho _{Gas} = \frac{m^*_{Gas} - (m_{Kolben} - m_{Waagschale} - \rho_{Luft} V_A)}{V_I} \]
\end{document}


\section*{Aufbau und Ablauf}


\section*{Rohdaten}


\section*{Auswertung}


\section*{Fehlerrechnung}


\section*{Diskussion}